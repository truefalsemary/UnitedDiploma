\section{ВВЕДЕНИЕ}

В последние десятилетия туризм претерпел значительную трансформацию и стал неотъемлемым компонентом глобальных экономических процессов, активно способствуя не только экономическому развитию, но и обеспечению устойчивого социального и культурного взаимодействия между различными регионами и странами мира. В современной экономике туризм играет существенную роль в формировании ВВП, создании рабочих мест, стимулировании малого и среднего предпринимательства, а также в повышении уровня жизни населения и решении социально значимых задач, таких как преодоление социального неравенства и интеграция различных слоев общества.

Актуальность исследования внутреннего туризма обусловлена не только его экономическими, но и социокультурными аспектами, которые активно обсуждаются в научных кругах. Туризм как форма социокультурной коммуникации позволяет странам и регионам обмениваться опытом, традициями, инновациями и знаниями, что способствует укреплению взаимопонимания и стабильности на национальном и международном уровнях. В условиях глобализации туристические потоки выступают в роли своеобразных культурных мостов, способствующих формированию толерантности и взаимоуважения среди различных сообществ.
Внутренний туризм в Российской Федерации в последние годы особенно актуализировался под воздействием глобальных экономических и социальных изменений, таких как пандемия COVID-19 и международные санкции. Если ранее внутренний туризм воспринимался преимущественно как второстепенная сфера по отношению к международному туризму, то сегодня он занимает стратегическое место в экономике и активно поддерживается государством через различные программы и инициативы. Пандемия COVID-19, ограничившая международные путешествия, стала катализатором переориентации туристических потоков на внутренние направления. Это привело к возникновению новых туристических маршрутов, развитию местных достопримечательностей и инфраструктуры, а также формированию новой модели поведения путешественников, стремящихся получать качественные услуги внутри страны.

По данным Федеральной службы государственной статистики (Росстат), вклад туризма в ВВП России, несмотря на значительный спад в 2020 году, обусловленный пандемией, продемонстрировал устойчивую тенденцию к восстановлению и росту. Так, если в 2018 году туристическая индустрия составляла 2,7\% от ВВП, то к 2023 году показатель восстановился до 2,8\%. Важную роль в этом процессе сыграли не только государственные меры поддержки, но и инициативы на региональном уровне, направленные на повышение туристической привлекательности территорий, развитие инфраструктуры и продвижение локальных туристических продуктов.

Рост спроса на внутренний туризм подтверждается также увеличением общего объёма услуг, предоставляемых туристическими агентствами, туроператорами и другими профильными организациями. Согласно данным Росстата, за последние пять лет этот показатель вырос почти на 66\%, достигнув 285,9 млрд рублей в 2023 году по сравнению с 172,1 млрд рублей в 2018 году. Такая динамика свидетельствует о высокой востребованности качественных туристических продуктов, что усиливает конкуренцию на рынке и подталкивает компании к внедрению инновационных решений и повышению профессионального уровня сотрудников.

Одним из наиболее заметных трендов в современном туризме становится растущая потребность путешественников в индивидуализации и персонализации туристического опыта \cite{12}. Современные туристы предпочитают уникальные впечатления, ориентированные на их персональные предпочтения, физические возможности и социальные особенности. В связи с этим возрастает важность внедрения инновационных технологий, таких как мобильные приложения, которые позволяют легко планировать путешествия, учитывать индивидуальные потребности пользователей и обеспечивать высокое качество сервиса на всех этапах туристического опыта.
Именно на удовлетворение таких индивидуализированных потребностей ориентировано разрабатываемое в рамках дипломного проекта мобильное приложение «Путешествия по России». Данное приложение предназначено для широкого спектра целевых аудиторий, включая маломобильных граждан, семьи с детьми, пожилых людей, студентов и других категорий путешественников, и предлагает им персонализированный подход к организации и планированию поездок. Приложение будет способствовать формированию активного туристического сообщества, члены которого смогут обмениваться опытом, рекомендациями и создавать уникальный контент, повышая привлекательность и узнаваемость местных туристических направлений.

Целью настоящей работы является всестороннее обоснование и практическая реализация цифровой платформы «Путешествия по России», способной предложить персонализированный туристический опыт для широкого круга внутренних путешественников и тем самым стимулировать социально-экономическое развитие регионов страны.
Для достижения поставленной цели сформулированы четыре взаимосвязанные задачи, отражающие распределение ролей в проектной команде:
\begin{enumerate}
    \item \textbf{Бизнес-аналитика и стратегия монетизации} – исследовать рынок travel-tech, определить целевые сегменты, построить финансовую модель и разработать долгосрочную стратегию доходов на основе партнёрских комиссий и нативной рекламы.
    \item \textbf{Продуктовая концепция и UX/UI-дизайн} – сформировать пользовательские сценарии, разработать инклюзивный интерфейс и визуальную айдентику, обеспечивающие высокую вовлечённость и удобство взаимодействия.
    \item \textbf{Кроссплатформенная мобильная разработка} – создать полноценное приложение на Flutter для Android, iOS и RuStore, реализовав модули авторизации, построения маршрутов, офлайн-карт и социального обмена контентом.
    \item \textbf{Разработка масштабируемого серверного ядра} – спроектировать микросервисную архитектуру на Go, внедрить безопасные механизмы аутентификации и горизонтального масштабирования, настроить \\ CI/CD-конвейер и систему мониторинга для бесперебойной работы платформы.
\end{enumerate}

\noindent Комплексное выполнение перечисленных задач позволит вывести на рынок технологически совершенный и коммерчески жизнеспособный продукт, отвечающий современным запросам российских путешественников и способный к дальнейшему масштабированию.
