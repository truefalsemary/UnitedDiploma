\section{ГЛАВА 1 ОПРЕДЕЛЕНИЕ ТРЕБОВАНИЙ}

\subsection*{1.1 Исследование целевой аудитории}
\addcontentsline{toc}{subsection}{1.1 Исследование целевой аудитории}

Для максимально эффективной реализации и соответствия разрабатываемого продукта запросам целевой аудитории был проведен комплексный анализ потребностей потенциальных пользователей мобильных туристических приложений. Данный этап исследования включал в себя количественный онлайн-опрос и качественные персональные интервью, направленные на глубокое понимание ожиданий, предпочтений и требований потенциальных пользователей.

Первый этап исследования был осуществлён посредством онлайн-опро\\-са на платформе Яндекс.Формы, который охватил 120 участников, представляющих различные возрастные и социальные группы: студенты, сотрудники университетов, коллеги-разработчики и родственники авторов проекта. Такая диверсификация выборки позволила получить репрезентативные данные, учитывающие разнообразие взглядов и потребностей различных социальных групп, и обеспечила надёжность и валидность полученных результатов.

На основе собранных данных были выделены ключевые потребности и ожидания от мобильных туристических приложений:
\begin{itemize}
    \item Возможность выбора и создания маршрутов с учётом различных сложностей и ограничений (отметили более 85\% респондентов);
    \item Подробная навигация и удобство использования приложения (93\% опрошенных);
    \item Доступ к информации о малоизвестных и оригинальных точках по России от других путешественников (74\% респондентов);
    \item Возможность быстрого обмена маршрутами и опытом между путешественниками непосредственно в приложении (67\% респондентов);
    \item Индивидуальное планирование маршрутов по протяженности и продолжительности путешествий (81\% участников опроса).
\end{itemize}


\noindent Второй этап исследования представлял собой качественные персональные интервью с 10 респондентами, охватывающими различные возрастные и социальные группы. Цель интервью состояла в уточнении и дополнении количественных данных, полученных в ходе онлайн-опроса. В процессе бесед респонденты выразили пожелания относительно создания удобного и интуитивно понятного интерфейса приложения, что позволило бы им быстро адаптировать маршруты под собственные нужды. Участники интервью подчеркнули также значимость эмоционального удовлетворения от путешествий, возможность обмена впечатлениями с другими пользователями и интерес к открытию новых, малоизвестных локаций.

Также анализируя собранные данные, команда проекта сегментировала целевую аудиторию – пользователей сервиса – на несколько групп, объединенных сходными характеристиками и потребностями. Сегментация важна для более точной настройки продукта под разные категории клиентов. Ниже приведены основные сегменты и их особенности:
\begin{itemize}
    \item 	Семьи с детьми ценят безопасность и комфорт. Предпочитают маршруты умеренной сложности, с наличием инфраструктуры (туалеты, места для отдыха, детские развлечения). Им важно заранее знать, подходит ли маршрут для детей, есть ли детские комнаты в музеях и пр. Также семьи заинтересованы в особых местах, интересных для разных поколений, и в сокращении времени на логистику, чтобы дети не утомлялись.
    \item 	Люди с ограниченной подвижностью или инвалиды, а также пожилые с проблемами здоровья. Для них критически важна доступность: наличие пандусов, лифтов, отсутствие длинных пеших переходов по пересечённой местности. Они нуждаются в маршрутах небольшой протяженности, с подробной информацией о рельефе, покрытиях дорог, наличии скамеек для отдыха. Сегмент требует тщательной фильтрации объектов по параметру доступности и ищет сервис, который избавит от необходимости самостоятельно уточнять эти условия.
    \item 	Путешественники пенсионного и предпенсионного возраста. Обычно они предпочитают более спокойный темп маршрута и умеренную нагрузку. Ценят подробное описание и историческую информацию о местах (культурно-познавательный аспект), а также простоту интерфейса приложения. Им важно избегать чрезмерно сложных для физической формы активностей (например, долгих горных походов). Этот сегмент также может нуждаться в увеличенном шрифте или голосовых подсказках в навигации, учитывая возможные возрастные ограничения зрения/слуха.
    \item 	В группу активных туристов входят в основном молодые или опытные путешественники, нацеленные на активный отдых и новые впечатления. Они готовы к маршрутам высокой сложности: длительным пешим прогулкам, походам, нестандартным видам активности (например, треккинг, велосипедные туры). Для них ключевым является наличие уникальных мест в маршруте – они стремятся уйти от массовых туристических троп и исследовать что-то новое. Активные туристы также более склонны сами делиться своими маршрутами и отзывами, выступая генераторами контента сообщества. Технически подкованы, поэтому с энтузиазмом используют продвинутые функции приложения (навигацию, социальные фичи).
\end{itemize}

\noindent Результаты проведённых исследований подчёркивают высокую востребованность и значимость ключевых функций разрабатываемого приложения. В частности, выделены такие приоритетные функции, как персонализированный подбор маршрутов, фильтрация по уровню сложности, доступности инфраструктуры и временным рамкам путешествия, а также обеспечение активного социального обмена опытом и маршрутами между пользователями. Таким образом, реализация этих возможностей в мобильном приложении «Путешествия по России» позволит не только эффективно удовлетворить выявленные потребности целевой аудитории, но и обеспечить конкурентные преимущества на рынке туристических приложений.

\subsubsection*{User Story № 1} 
Как молодой человек в возрасте 20–25 лет, начинающий изучать туристический потенциал своей страны, я стремлюсь получить детализированную и структурированную информацию о достопримечательностях, как широко известных, так и менее популярных маршрутах. Мне необходимо, чтобы представленные сведения сопровождались высококачественными фотографиями и интерактивными картами, что позволит формировать чёткое представление о будущих путешествиях, минимизировать неопределённость и обеспечивать максимально комфортное планирование поездок.

Критерии приемки:
\begin{itemize}
    \item 	Доступ к карточке маршрута, содержащей развернутое текстовое описание ключевых объектов;
    \item 	Визуализация маршрутов посредством качественных фотографий, позволяющих оценить эстетику и особенности местности;
    \item 	Интерактивная карта с точной геолокацией точек интереса, предоставляющая возможность предварительного анализа маршрута и его логистической доступности.
\end{itemize}

\subsubsection*{User Story № 2} 
Как родитель, путешествующий с детьми младшего возраста (до 10 лет), я заинтересован в функционале, который позволит мне находить и планировать маршруты, адаптированные под особенности семейного отдыха. Для меня критически важно, чтобы система обеспечивала фильтрацию маршрутов по уровню сложности и наличию соответствующей инфраструктуры, включая зоны отдыха, объекты общественного питания, санитарные узлы и детские площадки. Данный функционал позволит минимизировать потенциальные неудобства в поездке, повысить уровень комфорта передвижения и способствовать формированию благоприятного эмоционального опыта путешествия для всей семьи.

Критерии приемки:
\begin{itemize}
    \item 	Возможность выбора уровня сложности маршрута (например, лёгкий, средний, повышенной сложности);
    \item 	Наличие фильтрации маршрутов с учетом семейной инфраструктуры и ключевых удобств;
    \item 	Детализированное описание маршрутов, включающее информацию о наличии зон отдыха, пунктов питания и объектов, ориентированных на потребности семей с детьми.
\end{itemize}


\subsubsection*{User Story № 3} 
Как пользователь с ограниченными физическими возможностями (например, человек, передвигающийся на инвалидной коляске), я нуждаюсь в доступной и детализированной информации о маршрутах с адаптированной инфраструктурой. Для меня критически важно заранее получать сведения о наличии пандусов, ровных дорожек, отсутствии крутых спусков и подъёмов, а также возможности безопасного передвижения. Дополнительно необходимо учитывать возможность задания максимальной дистанции маршрута, что позволит мне эффективно планировать поездку с учётом физических возможностей и комфортного передвижения.

Критерии приемки:
\begin{itemize}
    \item 	Наличие специализированного фильтра для поиска маршрутов, адаптированных для маломобильных граждан;
    \item 	Возможность установки предельной дистанции маршрута в километрах в соответствии с индивидуальными потребностями пользователя;
    \item 	Подробное описание доступности инфраструктуры, включая информацию о наличии пандусов, ширине дорожек, доступности общественного транспорта и наличии санитарных зон, адаптированных для маломобильных граждан.
\end{itemize}

\subsubsection*{User Story № 4} 
Как опытный турист и тревел-блогер, занимающийся изучением малоизвестных регионов России, я стремлюсь к созданию индивидуальных туристических маршрутов с возможностью их последующего распространения среди сообщества единомышленников. Для меня важно иметь инструменты, позволяющие формировать маршруты с детализированными текстовыми описаниями, качественным визуальным сопровождением в виде фотографий и точной географической привязкой к картографическим данным. В дополнение к этому мне необходимо взаимодействовать с другими пользователями платформы, получать обратную связь в формате оценок и комментариев, а также привлекать внимание к уникальным местам, представляющим культурную, историческую и природную ценность.

Критерии приемки:
\begin{itemize}
    \item 	Разработка интуитивно понятного и функционального инструмента для создания пользовательских маршрутов;
    \item 	Возможность загрузки высококачественных фотографий и детализированных описаний маршрутов и достопримечательностей;
    \item 	Интеграция механизмов взаимодействия пользователей через систему комментариев, рейтингов и оценок, обеспечивающая активное вовлечение сообщества в процесс обмена туристическим опытом.
\end{itemize}

\subsubsection*{User Story № 5} 
Как молодой пользователь социальных сетей (возраст от 18 до 30 лет), я стремлюсь к активному взаимодействию с туристическим сообществом и обмену опытом в цифровом формате. Для меня важно иметь доступ к отзывам и оценкам других пользователей, чтобы принимать информированные решения при выборе маршрутов. Я также хочу иметь возможность оставлять собственные комментарии, делиться впечатлениями, присваивать рейтинги маршрутам и быстро находить самые популярные маршруты, основанные на пользовательских рекомендациях. Этот функционал позволит не только улучшить качество туристического опыта, но и создать динамичное цифровое сообщество путешественников.

Критерии приемки:
\begin{itemize}
    \item Возможность оставлять комментарии и оценки маршрутов, обеспечивая двустороннюю коммуникацию между пользователями;
    \item Доступ к отзывам и оценкам других путешественников, что позволит принимать более осознанные решения при выборе маршрутов;
    \item Быстрый просмотр наиболее популярных маршрутов, ранжированных на основе пользовательских оценок и отзывов.
\end{itemize}

\noindent Данный анализ позволил выявить ключевые тренды и определить наиболее перспективные направления для разработки мобильного приложения туристической тематики.

\subsubsection*{User Story № 6} 
Как пользователь, работающий полный рабочий день и имеющий ограниченное количество свободного времени для планирования путешествий, я нуждаюсь в функционале, который позволит мне оперативно получать уведомления о появлении новых маршрутов и туристических мероприятий, доступных в выходные дни и расположенных в непосредственной близости от моего местоположения. Это позволит мне быстро принимать решения о коротких поездках на 1–2 дня, избегая длительных поисков информации и повышая эффективность планирования отдыха.

Критерии приемки:
\begin{itemize}
    \item Интеллектуальная система уведомлений, анализирующая предпочтения пользователя и его геолокацию для предоставления актуальных предложений;
    \item Автоматическое отображение дат проведения мероприятий и туристических маршрутов, доступных в ближайшие выходные;
    \item Возможность персонализации уведомлений в настройках профиля, позволяя пользователю выбирать их частоту, тематику и метод получения.
\end{itemize}


\subsubsection*{User Story № 7} 
Как пожилой пользователь (возраст 60 лет и старше), я заинтересован в получении персонализированных рекомендаций по маршрутам, которые соответствуют моим культурным интересам (например, экскурсионные программы, посещение исторических и природных достопримечательностей) и физическим возможностям (маршруты с минимальной сложностью, адаптированные для комфортного передвижения). Мне важно избегать чрезмерных физических нагрузок, чтобы сосредоточиться на получении положительных впечатлений от путешествий без дополнительных затруднений.

Критерии приемки:
\begin{itemize}
    \item Автоматическая система персональных рекомендаций, основанная на истории путешествий пользователя и его предпочтениях;
    \item Возможность фильтрации маршрутов по уровню сложности и общей протяжённости, обеспечивающая комфортные условия передвижения;
    \item Доступ к рекомендованным маршрутам в отдельном разделе приложения, упрощающий навигацию и планирование поездок.
\end{itemize}


\subsubsection*{User Story № 8} 
Как студент, часто путешествующий в компании друзей-ровесников, я стремлюсь к удобной системе хранения и группировки избранных маршрутов в своём профиле приложения. Это позволит мне быстро возвращаться к планированию путешествий, сохранять наиболее интересные маршруты и эффективно делиться ими с друзьями. Такой функционал обеспечит гибкость в организации поездок и облегчит совместное обсуждение туристических планов.

Критерии приемки:
\begin{itemize}
    \item Возможность добавления маршрутов в избранное для быстрого доступа;
    \item Функция группировки маршрутов в тематические коллекции для систематизации информации;
    \item Инструменты удобного обмена избранными маршрутами с друзьями, способствующие коллективному планированию путешествий.
\end{itemize}


\subsubsection*{User Story № 9} 
Как администратор/менеджер туристического сообщества или организатор мероприятий, я нуждаюсь в доступе к аналитическим данным, отражающим популярность туристических маршрутов, отзывы пользователей, количество просмотров и посещений. Данный функционал позволит мне выявлять наиболее востребованные направления, анализировать предпочтения аудитории и на основе этих данных принимать обоснованные решения по разработке новых туристических программ и событий, ориентированных на актуальные интересы пользователей.

Критерии приемки:
\begin{itemize}
    \item Генерация наглядных отчётов и статистики по популярности маршрутов;
    \item Доступ к данным об отзывах, рейтингах и оценках маршрутов;
    \item Возможность анализа количества просмотров, посещений и пользовательских предпочтений в удобной веб-панели администратора, обеспечивающей эффективное управление туристическим контентом.
\end{itemize}

\subsubsection*{User Story № 10} 
Как пользователь, совершающий путешествия в регионы с нестабильным подключением к интернету (например, малонаселённые или труднодоступные территории России), я нуждаюсь в функционале офлайн-доступа к маршрутам и картам. Для меня критически важно иметь возможность заранее загружать необходимую информацию, включая детализированные описания маршрутов и навигационные данные, чтобы не зависеть от мобильного интернета во время поездки. Это обеспечит уверенность в ориентировании на местности и позволит избежать возможных сложностей, связанных с отсутствием связи.
Критерии приемки:
\begin{itemize}
    \item 	Функция предварительной загрузки маршрутов и карт для автономного использования;
    \item 	Беспрепятственный доступ к контенту приложения в офлайн-режиме без потери функциональности;
    \item 	Интуитивно понятный интерфейс для управления загрузкой и доступом к сохранённому офлайн-контенту.
\end{itemize}


\subsection*{1.2 Приоритизация функциональности}
\addcontentsline{toc}{subsection}{1.2 Приоритизация функциональности}

В результате детализированного анализа пользовательских историй команда разработчиков сформировала многоуровневую систему приоритизации функциональности приложения. Цель данного процесса заключается в оптимальном распределении ресурсов разработки, последовательном внедрении наиболее значимых функций и формировании стратегического видения эволюции продукта. Такой подход обеспечивает долгосрочную конкурентоспособность, соответствие запросам различных групп пользователей и адаптацию к изменяющимся требованиям туристического рынка.

\subsubsection*{Высокий приоритет (MVP — ключевой функционал первой версии)}
Функциональные возможности, относящиеся к данной категории, формируют базовую основу мобильного приложения. Их наличие является критически важным для первичного релиза, так как именно они обеспечивают удовлетворение базовых потребностей пользователей и определяют ценность продукта.

Основные элементы MVP:
\begin{itemize}
    \item 	Каталог маршрутов с подробными описаниями, визуальными материалами и картографической интеграцией. Этот элемент представляет собой ключевой механизм взаимодействия пользователей с сервисом, предоставляя исчерпывающую информацию о доступных маршрутах и способствуя информированному выбору туристических направлений.
    \item 	Гибкая система фильтрации маршрутов по уровню сложности, протяжённости и наличию инфраструктуры. Этот функционал ориентирован на удовлетворение потребностей широкого круга пользователей, включая семьи с детьми, пожилых путешественников и маломобильных граждан, для которых вопросы доступности играют решающую роль.
    \item 	Создание и публикация пользовательских маршрутов с интерактивными элементами. Данный функционал направлен на стимулирование пользовательского контента, вовлечение тревел-блогеров и опытных туристов, предоставляя им платформу для обмена знаниями и популяризации новых направлений.
\end{itemize}


User Stories: № 1 (каталог маршрутов), № 2 и № 3 (фильтрация маршрутов), № 4 (создание пользовательского контента).

\subsubsection*{Средний приоритет (улучшение пользовательского опыта после запуска MVP)}
Функции данного уровня приоритета направлены на повышение удобства взаимодействия с приложением, увеличение вовлечённости аудитории и укрепление пользовательской лояльности.
Функции среднего приоритета:
\begin{itemize}
    \item 	Механизмы социальной вовлечённости. Возможность оставлять комментарии, выставлять рейтинги и просматривать отзывы способствует формированию сообщества пользователей и усилению доверия к информации в приложении.
    \item 	Система персонализированных уведомлений. Этот функционал позволяет пользователям оперативно получать актуальные сведения о новых маршрутах, туристических мероприятиях и событиях на основе их предпочтений и геолокации.
    \item 	Сохранение избранных маршрутов и функция их обмена. Данная возможность особенно востребована среди пользователей, путешествующих в группах, а также среди тех, кто планирует поездки заранее и хочет иметь быстрый доступ к сохранённым маршрутам.
\end{itemize}

User Stories: № 5 (социальные взаимодействия), № 6 (уведомления), № 8 (избранные маршруты).

\subsubsection*{Низкий приоритет (перспективные функции для долгосрочного развития)}
Функциональность этой категории включает элементы, которые могут быть внедрены на более поздних стадиях развития продукта в зависимости от пользовательского спроса и общей стратегии масштабирования.

Ключевые направления:
\begin{itemize}
    \item 	Персонализированные рекомендации маршрутов, основанные на анализе пользовательского поведения и истории поездок.
    \item 	Оффлайн-доступ к маршрутам и картам для автономного использования в зонах с низким уровнем интернет-покрытия.
    \item 	Расширенные инструменты аналитики для организаторов туристических мероприятий и владельцев бизнеса.
    \item 	Продвинутая социальная интеграция, включая расширенные профили, возможность добавления друзей и интеграцию с внешними туристическими сервисами.
\end{itemize}

User Stories: № 7 (персонализация маршрутов), № 9 (аналитика), № 10 (офлайн-доступ).

Разработанная стратегия приоритизации функционала позволяет команде сосредоточиться на реализации наиболее значимых для пользователей возможностей, обеспечивающих успешный запуск продукта. Средне- и низкоприоритетные функции планируются к внедрению по мере роста пользовательской базы и накопления аналитических данных. Такой подход способствует устойчивому развитию приложения, повышению его востребованности и закреплению позиций на рынке цифровых туристических решений.


\subsection*{1.3	Оценка рынка}
\addcontentsline{toc}{subsection}{1.3	Оценка рынка}


В последние годы внутренний туризм в России демонстрирует значительный рост. В 2024 году количество туристических поездок по России составило около 92 миллионов, что является рекордным показателем за всю историю внутреннего туризма. По данным Российского союза туриндустрии, количество поездок по стране в 2024 году оценивается примерно в 96 миллионов[8], что на четверть больше, чем в 2023 году. Аналогичные данные приводит другой источник, указывая на 96 миллионов поездок в 2024 году, что на 25\% превышает показатель в 78 миллионов поездок в 2023 году. Согласно исследованию "СберАналитики", внутренний турпоток в России в 2023 году составил 152 миллиона человек, что является максимальным значением за последние пять лет. Другие данные указывают, что россияне совершили 160 миллионов поездок в 2023 году, и прогнозируется увеличение до 170 миллионов поездок в 2024 году. По данным другого источника, количество внутренних туристских перемещений достигло 108 миллионов в 2023 году, показав годовой прирост на уровне 11\%. Предварительные данные Российского союза туриндустрии (РСТ) также говорят о росте внутреннего турпотока на 20\% до 78 миллионов поездок в 2023 году. Таким образом, наблюдается устойчивая тенденция к увеличению количества внутренних туристических поездок в России. Разница в абсолютных значениях, вероятно, связана с различиями в методиках подсчета, включая определение понятия "турист" и "поездка" (например, учет краткосрочных поездок, посещений вторых домов и т.д.). Тем не менее, все источники сходятся во мнении о значительном росте рынка.

Помимо количества поездок, важным показателем является объем рынка в денежном выражении. В 2023 году туристическая отрасль России оказала услуг более чем на 4,3 триллиона рублей, что в 1,5 раза больше, чем в допандемийном 2019 году (2,8 триллиона рублей), причем основной прирост обеспечил внутренний туризм. Оборот внутреннего туризма за лето 2024 года вырос на 24\% и составил 917 миллиардов рублей[9], при этом прогнозируется, что годовой оборот может достичь 2-2,2 триллиона рублей. За первые девять месяцев 2023 года выручка туристического бизнеса в стране увеличилась примерно до 8,5 миллиардов рублей, что также в 1,5 раза больше показателей 2019 года. По данным Росстата, в 2023 году туристическая отрасль произвела товаров и услуг на 4304,7 миллиарда рублей по сравнению с 2789,9 миллиарда рублей в 2019 году. Доходы коллективных средств размещения за 11 месяцев 2024 года превысили 1 триллион рублей, что на 30\% больше, чем за аналогичный период 2023 года (256 миллиардов рублей). Эти данные свидетельствуют о значительном объеме рынка внутреннего туризма в России и его динамичном росте в денежном выражении. Увеличение расходов туристов говорит о готовности потребителей тратить на путешествия внутри страны.

Вклад туризма в экономику России также является важным показателем. В 2023 году доля валовой добавленной стоимости туристской индустрии в валовом внутреннем продукте Российской Федерации составила 2,8\%, вернувшись к уровню 2019 года. По другим данным, вклад туризма в экономику РФ составляет 3,47\% ВВП, или 3 триллиона рублей. Президент Российской Федерации поставил задачу увеличить долю туризма в ВВП до 5\%. Эти данные подчеркивают экономическую значимость туристического сектора для страны и потенциал для его дальнейшего развития и увеличения вклада в экономику.

По оценкам, около 3,3 миллиона человек с инвалидностью путешествуют в России. Общее количество людей с инвалидностью в России на начало 2023 года составляло 10,9 миллиона человек, или 7,5\% от общей численности населения. Возможность фильтрации маршрутов по сложности и протяженности делает приложение привлекательным для этого значительного и часто недостаточно обслуживаемого сегмента рынка.
Семьи с детьми составляют значительную долю туристического рынка. В летний сезон 2023 года на них приходилось 39\% всех продаж туров, а зимой 2023–2024 годов - 26\% [11]. По другим данным, около 30\% туристов путешествуют с детьми, а в 2024 году эта доля превысила 36\%. Учитывая, что летний сезон является пиковым для туроператоров и составляет 60–75\% годовых продаж, семьи с детьми являются ключевым целевым сегментом. 

Функциональность приложения, позволяющая выбирать маршруты с учетом потребностей детей (продолжительность, сложность), делает его ценным инструментом для этой аудитории.
Доля населения старше 55 лет в России достигла рекордных 30\%. Наблюдается растущая тенденция путешествий среди пенсионеров внутри страны, где в первой половине 2018 года 71\% их авиаперелетов приходилось на внутренние направления. В 2018 году 19\% пенсионеров в возрасте 60–75 лет выражали желание совершить туристическую поездку по России или за границу, и доля путешествующих внутри страны растет. В 2023 году 63\% респондентов старше 55 лет совершали поездки. Возможность выбора маршрутов с учетом физических возможностей делает приложение привлекательным для пожилых путешественников.
Почти каждый третий житель России (27\%) интересуется активным отдыхом. Наблюдается рост предпочтений к активному отдыху с 23\% до 29\%, особенно среди мужчин (39\%) и молодых людей до 34 лет (47\%). Приложение, позволяющее планировать пешие, велосипедные и другие активные маршруты, может привлечь эту значительную часть рынка.

Рынок внутреннего туризма в России демонстрирует устойчивые темпы роста в последние годы. Количество поездок выросло на 10–25\% в 2023 и 2024 годах [10], а доходы также показали значительный рост. Позитивная динамика наблюдается как в количестве поездок, так и в доходах, что свидетельствует о здоровом и расширяющемся рынке, создавая благоприятные условия для приложения "Путешествия по России".

Правительственные инициативы и инвестиции, направленные на развитие туристической инфраструктуры, а также прогнозируемый рост внутреннего туризма на 2025 год, указывают на то, что эта положительная тенденция, вероятно, сохранится. Цель президента по значительному увеличению вклада туризма в ВВП и количества туристических поездок еще больше укрепляет этот прогноз. Сочетание сильного исторического роста и позитивных будущих прогнозов, обусловленных как рыночным спросом, так и государственной поддержкой, указывает на высокий темп роста рынка в обозримом будущем, что делает запуск соответствующего туристического приложения своевременным.
Итого, закладывая целью в течение 3 трёх лет достичь 3\% охвата целевого рынка, то мы получаем аудиторию в 500 тыс. пользователей.

Итого, закладывая целью в течение 3 трёх лет достичь 3\% охвата целевого рынка, то мы получаем аудиторию в 500 тыс. пользователей.

\begin{table}[h]
\centering
\begin{tabular}{| p{75mm} | p{25mm} | p{25mm} | p{25mm} |}
% |l|c|c|c|}
\hline
\textbf{Показатель} & \textbf{2023 (оценка)} & \textbf{2024 (оценка)} & \textbf{Рост (YoY)} \\
\hline
Количество внутренних туристических поездок & 78-160 млн & 92-170 млн & 11-25\% \\
\hline
Оборот внутреннего туризма (лето) & 740 млрд руб. & 917 млрд руб. & 24\% \\
\hline
Общий оборот туристической отрасли & 4,3 трлн руб. & 2-2,2 трлн руб. (прогноз) & ~10\% \\
\hline
Доходы средств размещения (11 мес.) & 256 млрд руб. & 1 трлн руб. & 30\% \\
\hline
\end{tabular}
\caption{Показатели туристической отрасли}
% \label{Таблица 1.1}
\end{table}

\subsection*{1.4 Исследование конкурентной среды}
\addcontentsline{toc}{subsection}{1.4 Исследование конкурентной среды}

Исследование конкурентной среды является важным этапом подготовки проекта, поскольку позволяет выявить ключевые тенденции и лучшие практики, уже реализованные на рынке мобильных туристических приложений. Анализ существующих решений предоставляет возможность определить их сильные и слабые стороны, а также обозначить ниши и перспективы, которые могут быть эффективно заняты новым приложением.

\subsubsection*{Анализ конкурентного сервиса Яндекс.Карты}
Яндекс.Карты представляют собой ведущий российский сервис картографических и навигационных услуг, разработанный и активно поддерживаемый компанией «Яндекс». С момента запуска данный сервис завоевал широкую популярность среди пользователей не только в Российской Федерации, но и в странах СНГ, благодаря высокой точности предоставляемых данных, удобству использования и продуманной интеграции с другими цифровыми продуктами компании.

Преимущества Яндекс.Карт:
\begin{enumerate}
    \item Высокая детализация и точность картографических данных;
    \item Развитая многофункциональность и интеграция с общественным транспортом;
    \item Постоянное обновление и актуализация информации;
    \item Активное участие пользователей в создании контента.
\end{enumerate}

\noindent Недостатки Яндекс.Карт:
\begin{enumerate}
    \item Ограниченные возможности персонализации туристических маршрутов;
    \item Слабо развитые социальные функции;
    \item Ограниченное покрытие малоизвестных и удаленных туристических направлений;
    \item Недостаточные инструменты расширенной фильтрации и поиска.
\end{enumerate}

\noindent Несмотря на выявленные ограничения, Яндекс.Карты остаются одним из ведущих навигационных решений в России и странах СНГ, эффективно удовлетворяя основные потребности пользователей в навигации и планировании стандартных туристических поездок. Вместе с тем, существующие пробелы в функционале по персонализации, социальному взаимодействию и охвату локальных нишевых направлений открывают перспективы для нового продукта – приложения «Путешествия по России». Данный продукт может дополнить существующие возможности Яндекс.Карт, предоставив путешественникам продвинутые инструменты для индивидуального и специализированного планирования маршрутов, а также улучшить покрытие и качество информации о менее популярных и удалённых туристических объектах России.

\subsubsection*{Анализ конкурентного сервиса TripAdvisor}
TripAdvisor представляет собой одну из наиболее авторитетных и востребованных платформ в сфере международного туризма, специализирующуюся на обмене опытом, рекомендациями и объективными оценками пользователей. Основанная в 2000 году, компания заняла лидирующие позиции на глобальном туристическом рынке, сформировав крупнейшую в мире базу пользовательского контента, посвящённого гостиницам, ресторанам, достопримечательностям и другим туристическим объектам.

Преимущества TripAdvisor:
\begin{enumerate}
    \item Обширная и постоянно пополняемая база данных. Платформа аккумулирует миллионы отзывов, комментариев и фотографий, предоставленных путешественниками со всего мира, что позволяет пользователям быстро ориентироваться в особенностях и качестве предоставляемых услуг и объектов туристической инфраструктуры.
    \item Международное присутствие и широкая географическая доступность. TripAdvisor доступен на множестве языков и охватывает практически все туристически значимые регионы и страны мира. Это обеспечивает пользователям доступ к разнообразной и разносторонней информации, необходимой как для внутреннего, так и международного туризма.
    \item Высокая степень вовлечённости аудитории. Благодаря активной пользовательской базе, регулярно публикующей отзывы и мультимедийные материалы, платформа отличается высокой актуальностью контента и оперативностью обновления информации, что существенно облегчает принятие туристических решений.
    \item Комплексная интеграция с сервисами бронирования. Платформа обеспечивает возможность прямого бронирования гостиниц, ресторанов, экскурсий и мероприятий, упрощая тем самым процесс планирования и организации путешествий, увеличивая пользовательский комфорт и удобство.
\end{enumerate}

\noindent Недостатки TripAdvisor:
\begin{enumerate}
    \item Неоднородность качества и низкая модерация контента. В связи с большим объемом генерируемой пользователями информации возникают проблемы в обеспечении её достоверности, объективности и структурированности. Часто пользователи сталкиваются с субъективными или эмоционально окрашенными отзывами, которые могут создавать искажённое восприятие действительности.
    \item Недостаточные возможности персонализации маршрутов. Несмотря на обширность информационной базы, TripAdvisor ограничен в инструментах, позволяющих формировать и адаптировать маршруты с учетом специфических потребностей отдельных групп пользователей, таких как маломобильные граждане, семьи с детьми, пожилые туристы или лица с ограниченными возможностями.
    \item Ограниченность контента о локальных и малоизвестных направлениях. Особенно выражено это в отношении регионов, не обладающих широкой популярностью или удаленных от традиционных туристических центров. Это приводит к недостаточной информированности путешественников и сложности планирования поездок в подобные места.
    \item Ограниченные возможности фильтрации по специализированным критериям. TripAdvisor предлагает недостаточно гибкие инструменты фильтрации и поиска туристических объектов и услуг по специализированным параметрам, таким как уровень доступности для лиц с ограниченными возможностями, наличие семейной инфраструктуры или образовательный потенциал места.
    \item Избыточная коммерциализация и наличие рекламного контента. Активная коммерческая политика платформы, связанная с продвижением платных услуг и объектов, зачастую приводит к предвзятой расстановке акцентов и искажению результатов поиска и ранжирования, что может создать неудобства пользователям и снижать объективность оценки.
\end{enumerate}

\noindent TripAdvisor является важным игроком на мировом туристическом рынке, предоставляя обширный и востребованный функционал, который охватывает основные потребности путешественников. Вместе с тем, выявленные ограничения, такие как недостаток персонализированного подхода и слабое покрытие локальных туристических направлений, создают значительные перспективы для разработки и продвижения специализированных решений. Приложение «Путешествия по России» способно занять свою нишу на рынке, предложив пользователям качественно новые возможности персонализированного планирования маршрутов, детализированной фильтрации по специфическим критериям и улучшенного контента по региональным российским туристическим направлениям.

\subsubsection*{Анализ конкурентного сервиса Russpass}
Russpass является специализированной цифровой туристической платформой, реализуемой при поддержке Правительства Москвы с целью стимулирования развития и популяризации внутреннего туризма в Российской Федерации. Запущенная в 2020 году, платформа представляет собой значимый шаг в области цифровизации туристических услуг, предоставляя пользователям широкий спектр возможностей для удобного и эффективного планирования туристических поездок.

Преимущества Russpass:
\begin{enumerate}
    \item Комплексность и интегрированность предлагаемых услуг. Russpass предлагает пользователям удобные инструменты для организации полного цикла путешествия, включая бронирование гостиниц и других мест размещения, приобретение билетов на различные виды транспорта, заказ экскурсий, а также планирование развлекательных и культурных мероприятий. Такая комплексная интеграция различных услуг позволяет минимизировать затраты времени на организационные вопросы и повысить удобство путешествия.
    \item Государственная поддержка и сотрудничество с региональными администрациями. Благодаря тому, что платформа Russpass является проектом, инициированным Правительством Москвы, она активно взаимодействует с органами власти регионов, местными туристическими организациями и учреждениями культуры. Это взаимодействие обеспечивает высокий уровень достоверности и актуальности предоставляемой информации, способствует развитию региональной туристической инфраструктуры, привлекая дополнительные инвестиции и стимулируя развитие локальных туристических продуктов.
    \item Многоязычность и международная доступность. Russpass представлен на нескольких языках: русском, английском, испанском и арабском, что значительно расширяет целевую аудиторию и делает платформу привлекательной как для российских туристов, так и для гостей из-за рубежа. Такая многоязычность также способствует продвижению Российской Федерации как привлекательного туристического направления на международном уровне.
    \item Интуитивность интерфейса и удобство эксплуатации. Платформа Russpass характеризуется высоким уровнем удобства и простоты пользовательского интерфейса, что позволяет эффективно использовать её как опытным, так и начинающим путешественникам. Сервис регулярно улучшается и дорабатывается, обеспечивая простые и доступные инструменты для различных категорий пользователей.
\end{enumerate}

\noindent Недостатки Russpass:
\begin{enumerate}
    \item Ограниченность персонализации туристических продуктов. Несмотря на достаточно широкий ассортимент услуг, Russpass не обладает достаточным уровнем персонализации маршрутов, особенно когда речь идет о специальных потребностях отдельных категорий туристов, таких как маломобильные граждане, семьи с детьми, пожилые путешественники и лица с особыми потребностями здоровья.
    \item Региональная неравномерность качества и наполненности контента. Качество и разнообразие предложений платформы во многом определяются уровнем активности и заинтересованности региональных партнёров, что приводит к значительным различиям в представлении информации о разных регионах. Это может существенно ограничивать привлекательность платформы для путешествий в менее развитые туристические направления.
    \item Слабое развитие социального взаимодействия и коммуникационных функций. В отличие от более специализированных туристических сервисов, Russpass не обладает достаточными инструментами для активного взаимодействия пользователей между собой, такими как обмен рекомендациями, формирование сообщества путешественников и совместная организация поездок.
    \item Невысокая узнаваемость и необходимость усиленного маркетингового продвижения. Несмотря на поддержку на государственном уровне, узнаваемость Russpass среди широкой аудитории остаётся на относительно невысоком уровне, что требует дополнительных усилий и вложений в маркетинг и продвижение бренда.
\end{enumerate}

\noindent Russpass представляет собой важный проект, способствующий развитию внутреннего туризма в России посредством цифровизации и интеграции туристических услуг. Вместе с тем, выявленные недостатки и ограничения платформы открывают пространство для развития дополнительных специализированных решений. В этом контексте мобильное приложение «Путешествия по России» способно дополнить Russpass, предлагая туристам более глубокие возможности для персонализации маршрутов, увеличенное покрытие региональных направлений и развитые механизмы социального взаимодействия, что в итоге повысит привлекательность и конкурентоспособность туристических предложений.

\subsubsection*{Анализ конкурентного сервиса Komoot}
Komoot – это специализированная цифровая платформа для планирования маршрутов и навигации, ориентированная на активный отдых и туризм, включая пешие, велосипедные и туристические маршруты. Сервис был основан в 2010 году в Германии и активно развивается, привлекая аудиторию, увлечённую outdoor-активностями.

Преимущества Komoot:
\begin{enumerate}
    \item Подробное и интерактивное планирование маршрутов. Платформа предоставляет возможность составлять маршруты, учитывая различные критерии: тип активности (велосипед, пеший туризм, бег и другие), физическую подготовку пользователей, уровень сложности маршрута, а также интересы и предпочтения.
    \item Широкая интеграция с устройствами. Komoot совместим с большинством популярных GPS-устройств, включая велокомпьютеры и умные часы (Garmin, Wahoo, Apple Watch и другие). Это позволяет пользователям легко и удобно использовать платформу как во время планирования маршрута, так и в процессе его прохождения.
    \item Социальные функции и пользовательский контент. Пользователи имеют возможность делиться созданными маршрутами, комментариями и фотографиями, образуя сообщество единомышленников, что значительно увеличивает вовлечённость и привлекательность сервиса.
    \item Возможность использования в оффлайн-режиме. Пользователи могут заранее скачивать маршруты и карты, что особенно актуально для путешествий в регионы с нестабильным интернет-соединением.
\end{enumerate}

\noindent Недостатки Komoot:
\begin{enumerate}
    \item Ограниченность бесплатной версии. Для полного доступа к функционалу пользователям необходимо приобретать доступ к отдельным регионам или оформлять подписку, что ограничивает возможности использования сервиса для некоторых категорий путешественников.
    \item Разное качество картографической информации по регионам. В отдалённых и менее популярных туристических зонах карты и маршруты могут быть менее детализированы, что снижает качество пользовательского опыта.
    \item Зависимость от пользовательского контента. Качество представленных маршрутов и рекомендаций может значительно варьироваться, так как зависит от активности и компетентности самих пользователей.
\end{enumerate}

\noindent Статистические данные и показатели:
\begin{itemize}
    \item Согласно данным за 2022 год, платформа Komoot насчитывает более 30 миллионов пользователей по всему миру.
    \item Пользователями было создано более 100 миллионов индивидуальных маршрутов, что подчеркивает популярность сервиса и высокий уровень вовлечённости.
    \item Средняя оценка приложения в магазинах App Store и Google Play составляет 4,6 балла из 5 возможных, что свидетельствует о высоком уровне удовлетворенности пользователей.
\end{itemize}

\noindent Komoot занимает важную нишу на рынке туристических и навигационных решений для активного отдыха, предоставляя пользователям удобные инструменты для планирования и прохождения маршрутов. Несмотря на существующие ограничения, такие как необходимость оплаты дополнительных регионов и вариативность качества контента, платформа продолжает расти и привлекать новых пользователей. Для дальнейшего укрепления своих позиций Komoot необходимо работать над повышением качества картографической информации в удалённых регионах, а также развивать механизмы проверки и модерации пользовательского контента.

\subsubsection*{Сравнительный анализ конкурентных решений}

Для систематизации полученных данных представим сравнительную таблицу ключевых характеристик анализируемых сервисов:

\begin{table}[h]
\centering
\small
\begin{tabular}{| p{35mm} | p{25mm} | p{25mm} | p{25mm} | p{25mm} |}
\hline
\textbf{Критерий} & \textbf{Яндекс.Карты} & \textbf{TripAdvisor} & \textbf{Russpass} & \textbf{Komoot} \\
\hline
Географический охват & Россия, СНГ & Глобальный & Россия & Глобальный \\
\hline
Основная функция & Навигация & Отзывы и бронирование & Туристическая платформа & Планирование маршрутов \\
\hline
Персонализация маршрутов & Низкая & Низкая & Средняя & Высокая \\
\hline
Социальные функции & Слабые & Сильные & Слабые & Сильные \\
\hline
Покрытие региональных направлений & Высокое & Неполное & Неполное & Высокое \\
\hline
Фильтрация по доступности & Отсутствует & Ограниченная & Ограниченная & Средняя \\
\hline
Оффлайн-доступ & Есть & Нет & Нет & Есть \\
\hline
Монетизация & Реклама & Комиссии, реклама & Гос. финансирование & Подписка \\
\hline
Целевая аудитория & Все категории & Все категории & Туристы по России & Активные туристы \\
\hline
\end{tabular}
\caption{Сравнительный анализ конкурентных решений}
\end{table}

\subsubsection*{Общие выводы по конкурентному анализу}

Проведённое исследование конкурентной среды позволяет сделать несколько ключевых выводов, определяющих позиционирование и стратегию развития проекта «Путешествия по России».

\textbf{1. Фрагментированность рынка и отсутствие комплексного решения}

Анализ показал, что существующие сервисы специализируются на отдельных аспектах туристического опыта: навигация (Яндекс.Карты), отзывы и бронирование (TripAdvisor), туристическая платформа (Russpass), планирование спортивных маршрутов (Komoot). Ни один из рассмотренных продуктов не предлагает комплексного решения для планирования персонализированных туристических маршрутов по России с учётом различных потребностей пользователей.

\textbf{2. Недостаточная персонализация и адаптивность}

Большинство анализируемых платформ демонстрируют ограниченные возможности персонализации контента под специфические потребности различных категорий туристов. Особенно выражен дефицит решений для маломобильных граждан, семей с детьми и пожилых путешественников. Это создаёт значительную рыночную нишу для специализированного продукта с развитой системой фильтрации и адаптации маршрутов.

\textbf{3. Неравномерность покрытия региональных направлений}

Международная платформа TripAdvisor демонстрирует слабое покрытие российских региональных направлений, особенно малоизвестных локаций. Российские сервисы (Яндекс.Карты, Russpass) сосредоточены на популярных туристических центрах, оставляя значительные пробелы в освещении уникальных региональных маршрутов.

\textbf{4. Ограниченные социальные функции и взаимодействие пользователей}

За исключением TripAdvisor и частично Komoot, большинство сервисов не обеспечивают эффективного взаимодействия между пользователями. Отсутствуют механизмы обмена маршрутами, коллективного планирования поездок и формирования сообществ путешественников, что снижает вовлечённость аудитории.

\textbf{5. Технологические ограничения и зависимость от интернета}

Многие платформы демонстрируют критическую зависимость от стабильного интернет-соединения, что ограничивает их применимость в отдалённых регионах России. Только Яндекс.Карты и Komoot предлагают полноценную оффлайн-функциональность.

\textbf{6. Конкурентные возможности для нового продукта}

Выявленные пробелы создают благоприятную конкурентную среду для приложения «Путешествия по России». Основные конкурентные преимущества могут быть достигнуты через:
\begin{itemize}
    \item Глубокую персонализацию маршрутов с учётом физических возможностей и предпочтений пользователей;
    \item Комплексное освещение региональных российских направлений, включая малоизвестные локации;
    \item Развитые социальные функции для формирования активного сообщества путешественников;
    \item Интеграцию оффлайн-функциональности для использования в условиях ограниченной связи;
    \item Специализированные инструменты для различных категорий туристов (семьи, пожилые, маломобильные граждане).
\end{itemize}

\noindent Таким образом, конкурентный анализ подтверждает существование значительного рыночного потенциала для специализированного туристического приложения, ориентированного на внутренний российский туризм и обеспечивающего персонализированный подход к планированию путешествий.

\subsection*{1.5 Бизнес-модель стартапа}
\addcontentsline{toc}{subsection}{1.5 Бизнес-модель стартапа}
В этом разделе будет разобрана бизнес-модель стратапа по шаблону Александра Остервальдера.

\subsubsection*{Сегменты клиентов}
Основной целевой сегмент проекта «Путешествия по России» — туристы, планирующие поездки по территории страны. В эту группу входят различные демографические и поведенческие подгруппы: молодые путешественники и студенты, семейные туристы, пары и пенсионеры, интересующиеся экскурсиями и отдыхом внутри России. Кроме того, выделяются сегменты по типу интересов (любители природы, культурно-познавательного туризма, активных развлечений) и по формату поездки (самостоятельное планирование, групповые туры, корпоративный туризм). Привлечение клиентов будет осуществляться как среди постоянных отпускников, так и среди тех, кто раньше преимущественно путешествовал за рубеж: текущие тенденции свидетельствуют о росте внутреннего туризма в России. Так, по оценке «Яндекс.Путешествий», объём российского рынка гостиничных услуг в 2023 г. составит порядка 785 млрд руб., и при развитии внутреннего туризма он может вырасти до ~1 трлн руб. к 2027 г. Это подтверждает перспективность платформы, ориентированной именно на российских путешественников.

\subsubsection*{Ценностные предложения}
Проект «Путешествия по России» предлагает пользователям единый онлайн-сервис для планирования и бронирования поездок внутри страны. В качестве ценностного предложения выделяются: широкий ассортимент туристических продуктов (отелей, транспорта, экскурсий) с упором на региональные достопримечательности России; качественный контент (маршруты, путеводители, инсайдерские советы), который помогает пользователям открывать новые места; удобство и экономия времени за счёт объединения поиска, сравнения и бронирования в одном приложении. Сервис также обеспечивает систему поиска и фильтраций (учёт предпочтений путешественника) и поддержку на всех этапах путешествия. Наконец, платформа ориентирована на демонстрацию преимуществ российских регионов и малознакомых туристических направлений, что выгодно отличает её от международных OTA и стимулирует развитие внутреннего туризма.

\subsubsection*{Каналы}
Доступ к клиентам планируется через цифровые каналы. Основным каналом дистрибуции будет официальное мобильное приложение стартапа, оптимизированное для поиска маршрутов. Для привлечения аудитории будут использоваться социальные сети (Instagram, ВКонтакте, «Одноклассники», Telegram-каналы) с публикацией вдохновляющего контента и экскурсий, а также e-mail-рассылки с персональными предложениями. Важным каналом станут партнёрства с блогерами и медиа в туристической сфере: публикации и рекламные материалы в популярных туристических онлайн-изданиях и на порталах, посвящённых путешествиям по России. В офлайне проект может представлять себя на туристических выставках и сотрудничать с региональными туристскими организациями, размещая информацию на сайтах региональных турфирм.

\subsubsection*{Взаимоотношения с клиентами}
Стартап стремится выстраивать доверительные и долгосрочные отношения с путешественниками. На платформе реализуется система самообслуживания: пользователю доступен личный кабинет, где он может сохранять избранные маршруты, оформлять повторные заявки и отслеживать историю бронирований. Для поддержания контакта используются автоматические уведомления и социальное взаимодействие между пользователя в виде реакций и комментариев. 

\subsubsection*{Потоки доходов}
Основными источниками дохода проекта являются партнёрские комиссии и нативная реклама. Во-первых, платформа взимает комиссию с партнёров за каждое совершённое через сервис бронирование (отелей, билетов, туров и т.д.). Это соответствует модели агентства (agency model) в OTA-бизнесе: компания не закупает инвентарь, а получает процент от продаж партнёров. Так, по опыту крупных онлайн-тревел-агентств, средний размер комиссии составляет примерно 15–30\% от стоимости услуги. Во-вторых, проект продаёт рекламные места в виде интегрированных рекламных материалов (нативной рекламы) на сайте и в приложении. Таким образом, доход формируется за счёт комиссий с бронирований и платы за размещение нативных промо-материалов от партнёров, а также за счёт возможных платных интеграций с внешними сервисами (модель CPA/CPC при подключении сторонних офферов).

\subsubsection*{Ключевые ресурсы}
Ключевым ресурсом проекта является собственная техническая платформа (мобильное приложение) с метапоиском и системой управления контентом. Не менее важны накопленные базы данных по отелям, транспорту, экскурсиям, а также контент (фотографии, описания маршрутов, видео) о туристических направлениях России. Ключевыми ресурсами также являются разработчики проекта, которые поддерживают работоспособность сервиса.

\subsubsection*{Ключевые виды деятельности}
Основные виды деятельности проекта включают разработку и поддержку цифровой инфраструктуры (разработка новых функций, поддержка сайта и приложения), а также постоянное обновление и расширение контента (добавление новых маршрутов и путеводителей). Ежедневная работа включает поиск и подключение партнёров (автоматы бронирований гостиниц, железных дорог, туроператоров), ведение переговоров о сотрудничестве и условиях комиссий. Маркетинг и продвижение – ещё одна ключевая задача: настройка онлайн-рекламы, ведение соцсетей, создание рекламных кампаний и промо-акций. Проект также уделяет внимание аналитике пользовательского поведения (для улучшения рекомендаций) и обслуживанию клиентов (ответы на запросы, разрешение спорных ситуаций при бронировании).

\subsubsection*{Ключевые партнёры}
Ключевые партнёрские категории и форматы взаимодействия:
\begin{itemize}
    \item Отели и средства размещения: заключение договоров на интеграцию системы бронирования (через API или канал-менеджеры) и взимание комиссии с каждого подтверждённого бронирования. Проект предлагает отелям дополнительный поток клиентов, а за счёт гарантированной загрузки снимает риски недозагрузки.
    \item Транспортные компании: авиакомпании, железные дороги, автобусные перевозчики. Интеграция с их системами бронирования (или подключение через GDS/агрегаторы) позволяет пользователям искать билеты в одном интерфейсе. С транспортными партнёрами также обсуживается комиссия с продажи билетов.
    \item Экскурсионные и туроператоры: поставщики экскурсионных программ, активного отдыха, туров (как массовых, так и индивидуальных). С ними реализуется механизм размещения туров на платформе и получения комиссии с продаж турпакетов. Проект может совместно с туроператорами формировать пакетные предложения «отель + экскурсия» или «маршрут + транспорт».
    \item Региональные туристические организации и медиа: взаимодействие с туристическими ведомствами и региональными турагентствами для продвижения направлений. Это может быть обмен контентом (спонсируемые статьи, совместные маркетинговые кампании) и размещение спонсорского материала о регионах. Такие партнёры помогают расширить аудиторию проекта и обогащают его контентную часть.
\end{itemize}

\noindent Основные статьи затрат включают:
\begin{itemize}
    \item Разработка и поддержка ИТ-платформы (включая разработку, хостинг, закупку облачных ресурсов и лицензий).
    \item Расходы на маркетинг и продвижение (онлайн-реклама, SEO-продвижение, SMM, участие в выставках, партнерские программы).
    \item Фонд оплаты труда команды (разработчики, маркетологи, специалисты по работе с клиентами и партнёрами, контент-менеджеры).
    \item Операционные и административные расходы (офисные затраты, бухгалтерия, юридическое сопровождение, плата за платежные системы).
    \item Расходы на привлечение и удержание партнёров (например, бонусы за высокие показатели продаж, участие в совместных акциях).
\end{itemize}

\subsection*{1.6 Экосистема монетизации}
\addcontentsline{toc}{subsection}{1.6 Экосистема монетизации}
Проект «Путешествия по России» формирует доходы через две взаимосвязанные составляющие: комиссионные с партнёров и нативную рекламу. Комиссионные платежи являются основой бизнес-модели: стартап действует по принципу OTA-агентства (agency model), то есть получает процент от стоимости каждой успешно совершённой транзакции. 

Например, при бронировании отеля или экскурсии сервис удерживает установленную долю (часто в диапазоне 10–30\%) от суммы сделки. Такой подход снижает финансовые риски для проекта: оплата производится только при фактической продаже, партнёр не несёт дополнительных издержек, а платформа мотивирована привлекать больше бронирований для увеличения дохода. В роли партнёров могут выступать отели, туроператоры, транспортные компании, страховые провайдеры и т.д.; с каждым из них оговариваются условия комиссии и формат взаимодействия (API-подключение или офферы по модели CPA/CPC).

Второй компонент монетизации – нативная реклама. Под нативной рекламой понимаются рекламные материалы, органично встроенные в пользовательский опыт платформы (например, спонсируемые статьи, рекомендованные туры или отели в списках контента, интерактивные карты достопримечательностей с брендированными подсказками). Такой формат позволяет рекламодателям (отелям, турфирмам, туристическим регионам) продвигать свои предложения без агрессивных баннеров. Исследования показывают, что пользователи предпочитают нативную рекламу стандартным баннерам: например, по данным FreakOut, путешественники охотнее взаимодействуют с органично встроенными объявлениями и уделяют им больше внимания. 

При этом нативные объявления в совокупности получают более высокий CTR, чем обычные баннеры. По данным Outbrain/Econsultancy, размещённая на премиальных площадках нативная реклама вызывает на 44\% больше доверия со стороны аудитории и обеспечивает на 21\% более высокую кликабельность, чем традиционные форматы [13]; конверсия (покупки) в таких объявлениях также выше примерно на 24\%. Кроме того, нативные объявления воспринимаются как менее навязчивые: по их словам, 95\% пользователей негативно реагируют на прерывающую рекламу, тогда как нативные форматы не разрывают поток контента. Таким образом, сочетание партнёрских комиссий и нативной рекламы обеспечивает мультиканальный поток дохода. Комиссионная модель гарантирует, что платформа зарабатывает напрямую при росте продаж партнёров, а нативные объявления повышают доход за счёт платных промо-блоков и спонсорского контента. Преимуществами выбранной схемы являются низкий риск для стартапа (оплата от партнёров только за результат) и высокая эффективность рекламы: нативные форматы органично дополняют сервис и лучше вовлекают туристов, что выгодно и площадке, и рекламодателям.

\subsection*{1.7 Финансовая модель}
\addcontentsline{toc}{subsection}{1.7 Финансовая модель}
Распределение инвестиций. Основные статьи начальных вложений – оплата труда команды, маркетинг и инфраструктура. Предлагаемая смета вложений (примерно):

\begin{table}[h]
\centering
\begin{tabular}{| p{95mm} | p{35mm} |}
\hline
\textbf{Статья расходов} & \textbf{Сумма, руб} \\
\hline
Разработка MVP (3 мес. зарпл.) & 1\,800\,000 \\
\hline
Налоги и взносы (≈30\%) & 540\,000 \\
\hline
Маркетинг (реклама запуска) & 1\,000\,000 \\
\hline
Серверы и инфраструктура (год) & 300\,000 \\
\hline
Публикация в App Store/GP & 10\,000 \\
\hline
Поддержка/администрирование & 100\,000 \\
\hline
\textbf{Итого} & \textbf{3\,750\,000} \\
\hline
\end{tabular}
\caption{Смета начальных вложений}
\end{table}

За счёт всех статей требуется примерно 3,5–4,0 млн руб. стартовых инвестиций. Поддержка включает первые месяцы работы (например, техподдержку, модерацию). Затраты на серверы зависят от нагрузки: для небольшого приложения достаточно порядка 20–30 тыс. руб./мес (используя облачные VM и СУБД), то есть ~300 тыс. в год.
Монетизация. Приложение бесплатное, доход формируется за счёт нативной рекламы и партнёрских комиссий. Реклама: предлагаются местные объявления (отели, транспорт, экскурсии) как нативный контент в маршрутном приложении. Допустимая ставка CPM (стоимость 1000 показов) на российском рынке сейчас в среднем высокая – порядка 150 руб. CPM (данные об общих ставках CPM в соцсетях за 2024 год – 138–182 руб.). При этом средний пользователь может видеть, например, несколько десятков показов рекламы в месяц. Если считать, что пользователь генерирует 30 показов рекламы в месяц, при CPM 150 руб. это ~4,5 рубля/мес, или ~54 руб./год от рекламы на одного MAU.
Комиссии партнёров: за бронирование жилья и туров обычно берут около 5–15\% от суммы заказа. В РФ, например, агрегаторы брони отелей платят партнёрам ≈8–10\% (Яндекс.Путешествия – 8\%, «Островок» – 6\%, брендовые программы — 5–8\%). Авиабилеты приносят низкую комиссию (~1–3\%), поезда и автобусы – ≈5–10\%. Пусть в среднем при умелом продвижении в приложении на одного активного пользователя приходится комиссия ~100–200 руб. в год (зависит от конверсии пользователей, доли бронирующих и среднего чека). Например, если 20\% MAU совершают бронирование отеля на 20 000 руб., при комисии 10\% получаем ≈400 руб. на этих 20\% пользователей, т.е. ~80 руб. в среднем на пользователя. Добавляя туры/перевозки – может дойти до ~150–200 руб. с пользователя в год.
Таким образом средний годовой доход на пользователя (ARPU) будем оценивать примерно в 200–300 руб./год (≈17–25 руб./мес), включающий ~50–100 руб. от рекламы и ~150–200 руб. от комиссий. С такими допущениями финансовая модель будет выглядеть следующим образом.
Прогноз доходов. Оценим доходы по годам. Предполагаем постепенный рост MAU и доли монетизирующихся пользователей:

\begin{table}[h]
\centering
\begin{tabular}{| p{20mm} | p{30mm} | p{35mm} | p{35mm} | p{25mm} |}
\hline
\textbf{Год} & \textbf{Среднее MAU} & \textbf{Доход рекламы, млн руб} & \textbf{Доход комиссий, млн руб} & \textbf{Всего, млн руб} \\
\hline
1 & 30\,000 & (30\,000×70 руб)≈2.1 & (30\,000×200 руб)≈6.0 & 8.1 \\
\hline
2 & 70\,000 & (70\,000×180 руб)≈12.6 & (70\,000×240 руб)≈16.8 & 29.4 \\
\hline
3 & 120\,000 & (120\,000×210 руб)≈25.2 & (120\,000×250 руб)≈30.0 & 55.2 \\
\hline
\end{tabular}
\caption{Прогноз доходов}
\end{table}

Примечание: в модели Year1 консервативно взяты ARPU рекламы ≈70 руб/год (54 руб от CPM + неточисленная прочая реклама), ARPU комиссии ≈200 руб/год. К 3-му году ARPU можно ожидать рост до 300–400 руб./год (увеличится конверсия и средний чек): так доход от рекламы и комиссий существенно вырастет. Эти цифры иллюстративны, демонстрируют порядок величин доходов: уже во 2–3 год при хорошей динамике MAU доходы могут существенно превышать расходы.
Точка безубыточности и окупаемость. Постоянные затраты после запуска – преимущественно зарплаты (≈600 000 руб./мес) + инфраструктура (~30 000) + маркетинг (например, 100–200 тыс. руб.) ≈800 000 руб/мес. При ARPU ≈200–250 руб/год (≈16–21 руб/мес) для безубыточности нужно приблизительно 40–50 тыс. MAU. Например, 800 000 ÷ 20 ≈ 40 000 человек. Это MAU, при котором выручка покрывает расходы. С точки окупаемости: если первоначально вложено ~3,75 млн руб, а после выхода MAU растёт до точки безубыточности за ~9–12 месяцев, то при дальнейшем росте выручки проект может выйти в прибыль к концу 2-го года. При наших допущениях (ARPU ~300 руб/год, MAU стабильно растёт до ~70–100 тыс.) инвестиции вернутся за ≈20–24 месяца.
Оценка рынка и пользовательский потенциал
\begin{itemize}
    \item Объём рынка. Ежегодно россияне совершают сотни миллионов поездок по стране. Даже в кризисных условиях внутренний туризм резко растёт – 78 млн турпоездок в 2023, ожидается более 90–96 млн в 2024. Отпуск с ночёвкой в отеле практикуют десятки миллионов человек. Учитывая, что проникновение смартфонов – ≈73\% и распространение мобильных сервисов, потенциальная аудитория приложения измеряется десятками миллионов. Даже 1\% от поездок (≈0,8–1,0 млн поездок) может приносить сотни тысяч активных пользователей.
    \item Конкуренция. Рынок travel-приложений насыщен: крупнейшие сервисы («Яндекс.Путешествия» – 47\% пользователей, «Суточно.ру» – 40\%, «Туту.ру» – 39\% опрошенных) уже прочно вошли в жизнь путешественников. При этом нет массовых приложений, специализирующихся именно на создании/обмене маршрутов с учётом доступности. Реально притянуть в первый год можно несколько десятков тысяч MAU (консервативная оценка) – при благоприятном маркетинге и фокусе на нишевые сегменты. Например, даже 0,5\% от потенциальной базы (здесь 1\% внутреннего турпотока – ~1 млн поездок в год) даст порядка 5–10 тыс. пользователей; агрессивные маркетинговые кампании могут увеличить эту цифру до 50–100 тыс. MAU в год.
    \item Привлечение пользователей. Стоимость привлечения клиента в сегменте «Туризм» по рынку – около 1 800 руб., что говорит о высоких расходах на рекламу. С учётом этого и высокой конкуренции важно оптимизировать продвижение (таргет в соцсетях, коллаборации с региональными турофисами, контент-маркетинг). Цену клика в VK/Instagram сейчас можно оценивать ~10–20 руб., а CPM – 150–200 руб. При разумном маркетинге первые месяцы можно заложить бюджет ~500–1000 тыс. руб. на привлечение MAU.
\end{itemize}
\noindent Например, если годовые расходы составляют ~1,5–2 млн руб, а ARPU ≈ 100 руб/год, то безубыточный MAU будет порядка 15–20 тыс. пользователей в месяц. Если же учесть всю инвестицию 3,75 млн руб как расходы за период, то при ARPU ~80 руб/год точка безубыточности достигается примерно при MAU ≈ 45–50 тыс. пользователей. Таким образом, с учётом наших допущений проект выйдет на безубыточность уже в конце первого года или начале второго. В дальнейшем, при росте MAU до 100 тыс. к третьему году и сохранении ARPU ≈100 руб/год, чистая прибыль существенно превысит постоянные расходы. По нашим прогнозам, маржинальность бизнеса к концу третьего года может достигать 30–40\% (при учёте только переменных и операционных затрат). Таким образом, вложенные 3,75 млн руб полностью окупятся, а внутренняя норма рентабельности инвестиций (ROI) к концу прогнозного периода будет на уровне 100\% и выше.

\subsection*{Выводы по главе}
В первой главе была выполнена всесторонняя подготовительная работа, заложившая фундамент для дальнейшей разработки мобильного приложения «Путешествия по России».
\begin{itemize}
    \item Проведён сравнительный анализ четырёх ключевых сервисов (Яндекс.Карты, TripAdvisor, Russpass, Komoot). Выявлены их сильные стороны — высокая детальность карт, развитая социальная база, комплексные услуги бронирования — и ограничения — неоднородность качества контента, недостаток персонализации маршрутов, слабая работа с нишевыми регионами. Полученные результаты показали, что на рынке остаётся незаполненная ниша приложения, ориентированного именно на внутренний туризм по России, с акцентом на персонализированные маршруты, социальное взаимодействие и доступность малоизвестных локаций.
    \item Комбинация количественного онлайн опроса (120 респондентов) и качественных интервью позволила выявить основные потребности разных сегментов пользователей: молодых путешественников, семей с деть\\-ми, лиц с ограниченной мобильностью, пожилых туристов, тревел блогеров и организаторов мероприятий. На этой основе сформулированы десять подробных user story, охватывающих функциональные сценарии от базового поиска маршрутов до аналитики для администраторов.
    \item Также с помощью модели MVP установлены три уровня приоритета:
    \begin{itemize}
        \item Высокий (MVP) — каталог маршрутов с картами и фото, фильтрация по сложности и инфраструктуре, создание пользовательских маршрутов.
        \item Средний — социальные функции (комментарии, рейтинги), персонализированные уведомления, «избранное» и обмен маршрутами.
        \item Низкий — оффлайн доступ, углублённая персонализация, расширенная аналитика, продвинутая социальная интеграция.
    \end{itemize}
    \item Такая градация обеспечивает фокус команды на критически значимых функциях для первого релиза, сохраняя при этом стратегический план эволюции продукта.
\end{itemize}
\noindent Таким образом, в главе 1 сформулированы четкие функциональные и нефункциональные требования к будущей системе, определены целевые сегменты пользователей и их ключевые сценарии, а также обоснованы конкурентные преимущества разрабатываемого приложения. Полученные выводы послужат методологической основой для проектирования архитектуры и реализации микросервисной платформы, обеспечив соответствие конечного продукта реальным потребностям рынка внутреннего туризма в России.




