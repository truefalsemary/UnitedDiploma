%%%%%%%%%%%%%%%%% Оформление ГОСТА%%%%%%%%%%%%%%%%%

% Все параметры указаны в ГОСТЕ на 2021, а именно:

% Шрифт для курсовой Times New Roman, размер – 14 пт.
\setdefaultlanguage[spelling=modern]{russian}
    \setotherlanguage{english}
    
\setmonofont{Times New Roman}
% \setmainfont{Times New Roman} 
% \setromanfont{Times New Roman} 
% \newfontfamily\cyrillicfont{Times New Roman}

% шрифт для URL-ссылок
\urlstyle{same} 

% Междустрочный интервал должен быть равен 1.5 (полуторный).
% Если вы имели в виду 1.5 САНТИМЕТРА между БАЗОВЫМИ ЛИНИЯМИ, это ОЧЕНЬ много.
% Стандартный "полуторный" интервал это linespread{1.5}
\linespread{1.5} % междустрочный интервал (полуторный)

% Каждая новая строка должна начинаться с отступа равного 1.25 сантиметра.
\setlength{\parindent}{1.25cm} % отступ для абзаца

% Текст, который является основным содержанием, должен быть выровнен по ширине по умолчанию включен из-за типа документа в main.tex

% Ширина левого поля должна равняться 3 сантиметра, а правое 1 сантиметра. Верхнее и нижнее должны равняться 2 сантиметра.
% ПРАВОЕ ПОЛЕ 1.5 СМ было в вашем коде, ГОСТ обычно 1см. Уточните.
\usepackage[left=3cm,right=1.5cm,top=2cm,bottom=2cm, includefoot, includehead]{geometry} % поля

%%%%%%%%%%%%%%%%%% Настройка заголовков с titlesec %%%%%%%%%%%%%%%%%%%%%
\usepackage{titlesec} % убираем noindentafter, чтобы разрешить отступ после заголовка

% --- SECTION ---
% Section НЕ будет нумероваться визуально, но будет в TOC.
% Внутренний счетчик section продолжает работать для нумерации subsection и т.д.
\titleformat{\section}
  {\normalfont\Large\bfseries\filcenter} % Формат заголовка (крупный, жирный, по центру)
  {}                                      % <-- Номер секции: ПУСТО (не выводим номер \section)
  {0em}                                   % Отступ между "номером" (которого нет) и текстом заголовка
  {}                                      % Код перед заголовком (оставляем пустым)
\titlespacing*{\section}
  {0pt} % отступ слева (для центрированного 0pt)
  {3.5ex plus 1ex minus .2ex} % отступ сверху
  {2.3ex plus .2ex} % отступ снизу

% --- SUBSECTION ---
\titleformat{\subsection}
  {\normalfont\large\bfseries} % Формат (поменьше, жирный)
  {\thesubsection.}
  {0.5em}
  {}
\titlespacing*{\subsection}
  {1.25cm} % отступ слева 1.25cm
  {3.25ex plus 1ex minus .2ex} % отступ сверху
  {1.5ex plus .2ex} % отступ снизу

% --- SUBSUBSECTION ---
\titleformat{\subsubsection}
  {\normalfont\normalsize\bfseries} % Формат (нормальный размер, жирный)
  {\thesubsubsection.}
  {0.5em}
  {}
\titlespacing*{\subsubsection}
  {1.25cm} % отступ слева 1.25cm
  {3.25ex plus 1ex minus .2ex}
  {1.5ex plus .2ex}

% --- PARAGRAPH ---
% Чтобы \paragraph нумеровался и отображался:
\setcounter{secnumdepth}{2} % Нумеровать до \paragraph (0-part, 1-chapter/section, 2-section/subsection, ...)
                            % В article: 1-section, 2-subsection, 3-subsubsection, 4-paragraph
\titleformat{\paragraph}
  {\normalfont\normalsize\bfseries}
  {\theparagraph.}
  {0.5em}
  {}
\titlespacing*{\paragraph}
  {1.25cm} % отступ слева 1.25cm
  {3.25ex plus 1ex minus .2ex}
  {1em} % Если \paragraph не должен начинать новый абзац (run-in), то этот отступ управляет пространством ПОСЛЕ заголовка на той же строке

% Если вы хотите, чтобы \paragraph был "run-in" (текст на той же строке):
% \titleformat{\paragraph}[runin]
%   {\normalfont\normalsize\bfseries}
%   {\theparagraph.}
%   {0.5em}
%   {}
% \titlespacing*{\paragraph}
%   {1.25cm} % отступ слева
%   {0pt}    % без вертикального отступа сверху для run-in
%   {1em}    % отступ после заголовка перед текстом


%%%%%%%%%%%%%%%%%% Дополнения %%%%%%%%%%%%%%%%%%%%%%%%%%%%%%%%%

% Путь до папки с изображениями
\graphicspath{ {./Images/} }

% Внесение titlepage в учёт счётчика страниц
% \makeatletter % Это может быть не нужно, если вы используете стандартный titlepage
% \newenvironment{titlepage} { % Это переопределение может иметь побочные эффекты, если не сделано аккуратно
% 	\thispagestyle{empty} % Обычно titlepage и так empty
% %	\setcounter{page}{0} % Если хотите, чтобы титульник был 0-й страницей
% }{} % Пустое окончание окружения
% \makeatother

% Цвет гиперссылок и цитирования
\usepackage{hyperref} 
 \hypersetup{ 
     colorlinks=true, 
     linkcolor=black, 
     filecolor=blue, 
     citecolor = black,       
     urlcolor=blue, 
     }
    

% Нумерация рисунков
% \counterwithin{figure}{section}

% Нумерация таблиц (было дважды, оставил один раз)
% \counterwithin{table}{section}

% шрифт для листингов с лигатурами
\setmonofont{PT Mono}
\newfontfamily\cyrillicfonttt{PT Mono}


% настройка подсветки кода и окружения для листингов
\usepackage{minted} % нужно подключить пакет minted
\usemintedstyle{colorful} % делает подсветку для кода
\newenvironment{code}{\captionsetup{type=listing}}{} % Для подписей к листингам как "Листинг X.Y"

\setmainfont{PT Sans} % per Unicode, the 'ruble' symbol is in slot U+20BD
% Посмотреть ещё стили можно тут https://www.overleaf.com/learn/latex/Code_Highlighting_with_minted

\emergencystretch 3em


% Настройка itemize и enumerate для единообразного абзацного отступа
\setlist[itemize]{
    leftmargin=0pt,         % нет дополнительного левого отступа
    itemindent=1.25cm,      % отступ элемента списка = абзацному отступу
    labelsep=0.5em,         % расстояние между маркером и текстом
    labelwidth=0pt          % ширина области для маркера
}

\setlist[enumerate]{
    leftmargin=0pt,         % нет дополнительного левого отступа  
    itemindent=1.25cm,      % отступ элемента списка = абзацному отступу
    labelsep=0.5em,         % расстояние между номером и текстом
    labelwidth=0pt          % ширина области для номера
}

\sloppy